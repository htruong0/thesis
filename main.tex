\documentclass[twoside,openright,frontopenright]{ip3thesis}
\input{preamble}
\usepackage{subfiles}


\begin{document}
\title{Machine learning high multiplicity matrix elements for electron-positron and hadron-hadron colliders}
% \subtitle{An optional subtitle}
\author{Henry Truong}
\researchgroup{Institute for Particle Physics Phenomenology}
\maketitlepage*

\begin{abstract}
%
	The LHC is a large-scale particle collider experiment
	collecting vast quantities of experimental data to study the fundamental
	particles, and forces, of nature. Theoretical predictions made with
	the SM can be compared with observables measured at experiments.
	These predictions rely on the use of Monte Carlo event generators to 
	simulate events which demand the evaluation of a matrix element.
	For high multiplicity processes this can take up a significant
	portion of the time spent simulating an event.
	In this thesis, we explore the usage of machine learning
	to accelerate the evaluation of matrix elements by introducing a
	factorisation-aware neural network model. Matrix elements are plagued with singular
	structures in regions of phase-space where particles become soft or collinear,
	however, the behaviour of the matrix element in these limits are well-understood.
	By exploiting the factorisation property of matrix elements in these limits,
	the model can learn how to best represent the approximation
	of the matrix elements as a linear combination of singular functions.
	We examine the application of the model to $e^{-}e^{+}$
	annihilation matrix elements at tree-level and one-loop level, as well
	as to leading order $pp$ collisions where the acceleration of event generation
	is critical for current experiments.
%
\end{abstract}

\disableprotrusion
\tableofcontents*
\listoffigures
\listoftables
\enableprotrusion

\begin{declaration*}
%
	The work in this thesis is based on research carried out in the Department of
	Physics at Durham University. No part of this thesis has been
	submitted elsewhere for any degree or qualification.

	Research presented in this thesis is based on joint work:
	\begin{itemize}
		\item Chapter \ref{chapter:fame1} is based on \cite{Maitre:2021uaa}: Daniel Ma\^{i}tre and Henry Truong, \textit{A factorisation-aware Matrix element emulator}, Journal of High Energy Physics, 2021 (11). 066.
		\item Chapter \ref{chapter:fame2} is based on research with Daniel Ma\^{i}tre in preparation to be published.
		\item Chapter \ref{chapter:fame3} is based on research with Timo Janßen, Daniel Ma\^{i}tre, Steffen Schumann, and Frank Siegert in preparation to be published.
	\end{itemize}

	Other research projects published during my studies but not included
	in this thesis:
	\begin{itemize}
		\item \cite{doi:10.1098/rsos.210506}: J. Aylett-Bullock, C. Cuesta-Lazaro, A. Quera-Bofarull, M. Icaza-Lizaola, A. Sedgewick, H. Truong, A. Curran, E. Elliott, T. Caulfield, K. Fong, I. Vernon, J. Williams, R. Bower and F. Krauss, \textit{June: open-source individual-based epidemiology simulation}, (2021), R. Soc. open sci.8210506210506.
		\item \cite{doi:10.1098/rsta.2022.0039}: I. Vernon, J. Owen, J. Aylett-Bullock, C. Cuesta-Lazaro, J. Frawley, A. Quera-Bofarull, A. Sedgewick, D. Shi, H. Truong, M. Turner, J. Walker, T. Caulfield, K. Fong and F. Krauss, \textit{Bayesian emulation and history matching of JUNE}, (2022), Phil. Trans. R. Soc. A.3802022003920220039.
	\end{itemize}

%
\end{declaration*}

\begin{acknowledgements*}
%
	The work carried out during this journey would not have been
	possible without all the help and support from the people around me.
	Firstly, I would like to express my gratitude to Michael Spannowsky, and
	to my supervisor Daniel Ma\^{i}tre, who gave me a chance to undertake this
	doctorate when nobody else did.
	Daniel has always been supportive with his meticulous advice and constructive criticisms,
	and I thank him for giving me the freedom to explore different avenues during our joint research projects.
	
	All the people at the IPPP, especially Trudy and Joanne, have fostered a comfortable working
	and social environment of which I was very glad to be a part of.
	To all the students who have come and gone during my time at the IPPP, it was a pleasure
	to get to know you all. Special mentions go to Alan, Asli, Dorian, Francesco, Joe, Kevin, Marian,
	Ryan, and of course all the wonderful people in OC215 and beyond.
	Outside of the particles, shout-out to Aidan, Cameron, Ed, Jack, and Vicky for being great friends
	who I could always have a laugh with and chat about anything. I want to say thank you to Jack
	in particular for being a friend I could always count on for a good time, and a serious time.
	
	I want to express my deepest gratitude to my dear family and friends back home. Mum, dad, Annie,
	and Tracy, Chris and Tyrone, your endless love and support has been more helpful than you can imagine
	in completing this journey.

	These past four years of my life have been tremendously formative in many ways. All the people
	I have met, I have learned so much from, and I will cherish all
	the memories that we shared together. I will leave Durham as a better person.
%
\end{acknowledgements*}

\cleardoublepage

\subfile{intro}
\subfile{qcd}
\subfile{MC}
\subfile{ml}
\subfile{paper1}
\subfile{paper2}
\subfile{paper3}
\subfile{conclusion}

\appendix
\subfile{appendix1}
\subfile{fame_appendex1}
%\subfile{appendix2}

\bibliographystyle{JHEP}
\bibliography{bibliography}

\end{document}
