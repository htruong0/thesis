\documentclass[main.tex]{subfiles}
\begin{document}
\chapter{Conclusion and outlook}
\label{chapter:conclusion}
In this thesis we have presented a physics-inspired
neural network model to emulate matrix elements. The
application of using a neural network to fit functions
is a common scenario, however, the unique singularity
structure of physical matrix elements precludes a naive
approach if accuracy is desired. Indeed, the per-point
accuracy of matrix elements is generally desired, especially
in the infrared regions of phase-space where the emergence of
large weights can contribute significantly to the total cross-section.
Fortunately, matrix elements factorise in these infrared
regions of phase-space, with well-understood explicit functional forms
describing the single soft and collinear limits.
By incorporating these functions into the neural network model via an
ansatz composed of a linear combination of these singular
functions, it is possible to transform the problem of fitting
a potentially rapidly changing function over phase-space
to the fitting of more well-behaved coefficients. We
dubbed this family of models factorisation-aware
as the neural network learns to choose the relevant
functional behaviours in specific soft and collinear limits
where the fitted coefficients are constrained. Outside
of these limits, we make use of the well documented fitting capabilities
of neural networks to form an interpolation.

We first demonstrated the use of the factorisation-aware
model in Chapter~\ref{chapter:fame1}
where we utilised the Catani-Seymour dipole functions as
the singular functions in the ansatz to model tree-level
electron-positron annihilation matrix elements. It was
shown that by building these functions into the emulator,
there was drastic improvements in per-point accuracy compared to
existing methods, with accuracy well below 1\% for all processes studied.
Furthermore, the inclusion of the singular
functions describing infrared behaviour of the matrix
elements allowed well-behaved extrapolation of the model
into untrained regions of phase-space which were potentially
more singular than those seen previously during training.

The success of this approach prompted the study of applying
the factorisation-aware model to emulate the more expensive
one-loop matrix elements for the same family of processes,
as seen in Chapter~\ref{chapter:fame2}.
It was found that by adapting the
Catani-Seymour dipole functions to the more relevant antenna functions
containing one-loop singular behaviour, the
accurate emulation of NLO QCD k-factors was achieved.
In this study we also demonstrated that it was possible
for the emulator to learn the renormalisation scale dependence
of the one-loop matrix elements, allowing it to be used to
carry out scale variations.
We showed that $e^{+}e^{-} \rightarrow q \bar{q} ggg$ NLO k-factors
could be predicted with accuracy at the 1\% level, representing a speed up in
evaluation of over 3 orders of magnitude compared to {\MadGraph} on
a single CPU core.

The study of electron-positron annihilation matrix elements
whilst important for planned future lepton-lepton colliders,
is not critical for making predictions for the LHC. For that,
hadron-hadron initiated matrix elements are much more relevant.
In Chapter~\ref{chapter:fame3}, we extended the factorisation-aware
model to hadronic collisions by incorporating the full set
of massless and massive dipole functions. The inclusion of these
functions into the emulation model allowed us to model $Z+\{4,5\}$ jets
and $t\bar{t}+\{3,4\}$ jets matrix elements at leading order, which
represent computationally expensive and phenomenologically relevant
processes. We found
that the accuracy of the emulator, while lower than in the cleaner
$e^{+}e^{-}$ environment, was not surprising given that accuracy
scales inversely with multiplicity and these were the highest
multiplicity processes considered in this thesis.
However, the well-behaved
predictions in the infrared limits and the low population tails
makes it suitable for use as a surrogate model for event
weights in the context of event unweighting, where stability
of predictions is desirable.

We showed that using the emulator in a novel two-stage
unweighting procedure, where small mismatch from the full event
weight is corrected for, could reduce the time spent in unweighting
a sample of events. We saw that the unweighting of events
could be accelerated by well over 2 orders of
magnitude for the most computationally expensive processes,
$Z+5$ jets and $t\bar{t}+4$ jets.

The factorisation-aware model presented in this thesis
has been demonstrated to be a powerful tool to emulate
matrix elements for a variety of processes, at
tree-level and one-loop level, owing to the
flexibility of the modelling philosophy.
The path forward is clear: to reliably accelerate the generation of
simulated unit-weight events required for LHC experiments, the
modelling procedure has to be extended to arbitrary SM
processes, both for colour-summed, and colour-sampled matrix
elements, with an easy-to-use interface to general
purpose Monte Carlo event generators.
We have taken the first steps towards this by proving the viability
of the model in a production setting by deploying it
within the {\Sherpa} framework to successfully accelerate the
process of unweighting events.

\end{document}