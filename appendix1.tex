\documentclass[main.tex]{subfiles}
\begin{document}
\chapter{Catani-Seymour notation}
\label{appendix:catani_seymour}
    This appendix serves as supplementary material for Chapter~\ref{chapter:qcd},
    specifically, for the tree-level matrix element factorisation in Section~\ref{sec:me_factorisation}
    and Catani-Seymour dipoles in Section~\ref{sec:CS_dipoles}.
    \section{Notation}
    The notation for matrix elements given in Ref.~\cite{Catani:1996vz}
    is repeated here for convenience.

    Consider a tree-level matrix element with $n$ partons in
    the final-state
    \begin{equation}\label{eqn:CS_ME}
        \mathcal{M}^{c_{1},\ldots,c_{n};s_{1},\ldots,s_{n}}_{n}(p_{1},\ldots,p_{n}) \, ,
    \end{equation}
    where $\{c_{1},\ldots,c_{n}\}$ are colour indices,
    $\{s_{1},\ldots,s_{n}\}$ are spin indices,
    and $\{p_{1},\ldots,p_{n}\}$ are momenta.

    Introducing the basis $\{| c_{1},\ldots,c_{n} \rangle \otimes | s_{1},\ldots,s_{n} \rangle \}$
    in colour + helicity space allows us to write
    \begin{equation}\label{eqn:CS_hilbert}
        \mathcal{M}^{c_{1},\ldots,c_{n};s_{1},\ldots,s_{n}}_{n}(p_{1},\ldots,p_{n}) \equiv \left(\langle c_{1},\ldots,c_{n} | \otimes \langle s_{1},\ldots,s_{n} | \right) | 1, \ldots,n \rangle_{n} \, ,
    \end{equation}
    where $|1, \ldots, n \rangle_{n}$ is a vector in colour + helicity space.
    The indices inside the vector denote the spin and colour indices,
    while the subscript denotes the multiplicity of the matrix element.
    The matrix element squared, summed over final-state colours and spins,
    can therefore be written as
    \begin{equation}\label{eqn:CS_ME2}
        |\mathcal{M}_{n}|^{2} = {}_{n} \langle 1,\ldots,n|1,\ldots,n\rangle_{n} \, .
    \end{equation}

    In the case of partons in the initial-state the colour + helicity vector
    $|1, \ldots, n \rangle$ has to be normalised by a factor $\sqrt{n_{c}}$ for each
    initial state parton. For hadron-hadron collisions this amounts to
    \begin{equation}
        | 1, \ldots, n \rangle_{n} \rightarrow \dfrac{1}{\sqrt{n_{c}(a)n_{c}(b)}} | 1, \ldots, n \rangle_{n} \, ,
    \end{equation}
    where $n_{c}(a)$ denotes the number of colour states of parton $a$.
    Namely $n_{c}(q) = n_{c}(\bar{q}) = N_{c}$ and $n_{c}(g) = N_{c}^{2}-1$.

    The colour-charge operators $\bm{T}_{i}$ act on colour space
    to give the colour-correlated matrix element
    \begin{equation}
        \label{eqn:charge_operator}
        \begin{split}
            {}_{n} \langle 1, \ldots, n ; a, b | \bm{T}_{I} \cdot \bm{T}_{J} &| 1, \ldots, n ; a, b \rangle_{n} = \\
            & \quad \dfrac{1}{\sqrt{n_{c}(a)n_{c}(b)}} \left[ \vphantom{\dfrac{1}{1}} \mathcal{M}_{n}^{c_{1},\ldots,c_{I},\ldots,c_{J},\ldots,c_{n}}(p_{1},\ldots,p_{n};p_{a},p_{b}) \right]^{*} \\
            &\times \mathcal{T}^{e}_{c_{I}d_{I}}\mathcal{T}^{e}_{c_{J}d_{J}} \times \left[ \vphantom{\dfrac{1}{1}} \mathcal{M}^{d_{1},\ldots,d_{I},\ldots,d_{J},\ldots,d_{n}}(p_{1},\ldots,p_{n};p_{a},p_{b})\right] \, ,
        \end{split}
    \end{equation}
    where $I$ and $J$ denotes initial- or final-state partons.
    For a parton $I$, the matrices $\mathcal{T}^{e}_{c_{I},d_{I}}$ are defined as
    \begin{equation}
        \mathcal{T}^{e}_{c_{I}d_{I}} =
        \begin{cases}
            -i f_{cde} \hspace{-6pt} \quad \text{if} \quad I = \text{gluon} \, , \\
            \hphantom{-} T^{e}_{cd} \quad \text{if} \quad I = \text{final-state quark or initial-state antiquark} \, , \\
            -T^{e}_{cd} \quad \text{if} \quad I = \text{final-state antiquark or initial-state quark} \, .
        \end{cases}
    \end{equation}
    Each colour vector $|1, \ldots, n; a, b \rangle$ is a colourless
    state, meaning colour conservation can be written as
    \begin{equation}
        \sum_{I} \bm{T}_{I} |1, \ldots, n; a, b \rangle = 0 \, .
    \end{equation}
    Additionally, the colour-charge operators have the following properties
    \begin{align}
        \bm{T}_{I} \cdot \bm{T}_{J} &= \bm{T}_{J} \cdot \bm{T}_{I} \quad \text{for} \quad I \neq J \, , \nonumber \\
        \bm{T}^{2}_{q} &= \bm{T}^{2}_{\bar{q}} = C_{F} \, ,\\
        \bm{T}^{2}_{g} &= C_{A} \, . \nonumber
    \end{align}
\end{document}