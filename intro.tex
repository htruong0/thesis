\documentclass[main.tex]{subfiles}
\begin{document}
\chapter{Introduction}
\label{chapter:intro}
The standard model of particle physics (SM) is our current best working
theory that describes the phenomenon we observe, for  example,
 at collider experiments.
The large hadron collider (LHC) is a high energy proton-proton
collider experiment that allows us to study the fundamental particles
laid out in the standard model.

The interaction of these fundamental particles are described
by scattering amplitudes. The standard model allows us to
calculate these amplitudes, or matrix elements, from first
principles. A key challenge in the past few decades has been
to compute these matrix elements for increasing number of
final state particles and for higher orders in perturbative
expansions in fixed order calculations. Although there
has been tremendous progress in this sector a complication
that naturally arises is that the evaluation of these matrix
elements becomes increasingly time-consuming due to their
algebraic complexity.

The topic of this thesis is to apply modern machine learning
techniques to reduce the time spent evaluating these
complex matrix elements. Considering the physical properties
of these matrix elements allows us to implement physics
knowledge into these machine learning algorithms giving
us higher levels of performance than would otherwise be
achievable.

The structure of this thesis is as follows: in Chapter~\ref{chapter:intro}
I will recap the necessary concepts from the standard
model, in particular, on quantum chromodynamics (QCD), the sector
which governs the behaviour of quarks and gluons. Furthermore,
the relationship between theory and experiment will be elaborated on.

In Chapter~\ref{chapter:qcd} I will discuss the challenges that arise
in fixed order perturbative calculations and the current methods
that have been adopted in the community to deal with them. These
methods also form the basis of our physics knowledge which we
embed into the machine learning algorithm. This procedure is 
described in detail in Chapter~\ref{chapter:fame1} for electron-positron
annihilation for tree-level matrix element emulation.
A similar philosophy is applied in Chapter~\ref{chapter:fame2}
where next-to-leading order QCD k-factors are emulated.
These two chapters discuss in detail the procedure for
electron-positron annihilation, however, the most relevant
experiment currently is the LHC which is colliding protons.
The extension to hadron-hadron colliders is detailed in
Chapter~\ref{chapter:fame3} where we explore using the
emulator in a novel implementation to replace traditional
matrix element providers in the event generator SHERPA.

Finally, I will conclude the thesis in Chapter~\ref{chapter:conclusion}.

\section{The Standard Model of particle physics}
    The Standard Model of particle physics (SM) is our current
    best working theory to describe all known elementary particles
    as well as three of the four fundamental forces. Developed
    predominantly in the latter half of the 20th century, it is
    one of the most well tested theories we have in science today.
    Some highlights include the highly precise predictions of the
    anomalous magnetic moment of the electron, agreeing with experimental
    measurements to more than 10 significant figures \cite{Aoyama:2017uqe},
    and the discovery of the Higgs boson in 2012 by the ATLAS \cite{ATLAS:2012yve}
    and CMS experiments \cite{CMS:2012qbp} at the Large Hadron Collider
    (LHC) which was theorised decades prior.

    The SM is a gauge quantum field theory (QFT) where particles
    are described as excitations in quantum fields. It can be specified
    by the gauge group
    \begin{equation}\label{eqn:SM_gauge}
        \mathrm{SU}(3)_{c} \times \mathrm{SU}(2)_{L} \times \mathrm{U}(1)_{Y}\, ,
    \end{equation}
    where subscripts denote the charges of the gauge groups. The first gauge
    group with colour charge $C$ describes the interactions of the
    strong force within the theory of quantum chromodynamics (QCD), which we
    will elaborate more on in the next section. The second
    and third gauge groups represent the electroweak sector of the SM with
    charges $L$ for left and $Y$ for hypercharge. Under electroweak
    spontaneous symmetry breaking (EWSB), this product becomes
    \begin{equation}\label{eqn:SM_SSB}
        \mathrm{SU}(2)_{L} \times \mathrm{U}(1)_{Y} \xrightarrow{\mathrm{EWSB}} \mathrm{U}(1)_{\mathrm{EM}} \, ,
    \end{equation}
    giving rise to the electromagnetic and weak forces we observe.
    The effect of gravity is considered to be negligible on the scales
    considered in the SM and so it is not described.
    Each of the three fundamental forces described by the SM is mediated
    by the exchange of a gauge boson. The gauge bosons mediating the strong
    and EM force are the massless gluon $g$ and massless photon $\gamma$,
    respectively. For the weak force the gauge bosons are the $W^{\pm}$
    and $Z^{0}$ bosons which attain a mass through the Higgs mechanism \cite{Englert:1964et,Higgs:1964pj,Guralnik:1964eu}
    during EWSB, which elucidates the Higgs boson $H$.

    The matter content of the SM consists of fermions which can be split
    into quarks and leptons. Quarks are massive and experience the strong,
    weak, and EM force. Leptons are defined by their lack of colour charge,
    meaning they do not experience the strong force. Leptons can be separated
    into charged leptons (electron $e$, muon $\mu$, tau $\tau$
    and their antiparticles) which experience the weak and EM force, and 
    neutrinos (electron neutrino $\nu_{e}$, muon neutrino $\nu_{\mu}$, tau
    neutrino $\nu_{\tau}$ and their antiparticles) which only experience the
    weak force. Neutrinos are massless in the SM but this is not a requirement
    of the model. In fact they have been observed to have mass \cite{Super-Kamiokande:1998kpq,SNO:2002tuh}.

    The particle content of the SM is summarised in Figure \ref{fig:SM_particles}
    showing the quarks, leptons and bosons along with their masses, charges
    and spin.
    \begin{figure}
        \includegraphics[width=\linewidth]{introduction/sm_particles.pdf}
        \caption{Particle content of the SM split into quarks, leptons,
        gauge bosons and scalar bosons. The columns for the fermions depict
        the three different generations. Masses, electric charge, and spin
        are given for each particle. The yellow contours indicate the coupling
        of bosons to fermions, illustrating the forces experienced by the
        fermions inside the contour. Figure reference \cite{SM_figure}.}
        \label{fig:SM_particles}
    \end{figure}
    The interactions of these fields are governed by the SM Lagrangian
    \footnote{technically Lagrangian density but we use the terms
    Lagrangian and Lagrangian density interchangeably}
    which can be written as
    \begin{equation}\label{eqn:L_SM}
        \mathcal{L}_{\mathrm{SM}} = \mathcal{L}_{\mathrm{gauge}} + \mathcal{L}_{\mathrm{fermion}} + \mathcal{L}_{\mathrm{Higgs}} + \mathcal{L}_{\mathrm{Yukawa}} + \mathcal{L}_{\mathrm{GF}} + \mathcal{L}_{\mathrm{ghost}} \, ,
    \end{equation}
    where $\mathcal{L}_{\mathrm{gauge}}$ describes the gauge fields,
    $\mathcal{L}_{\mathrm{fermion}}$ describes how fermions interact with
    gauge fields as well as their kinetic terms,
    $\mathcal{L}_{\mathrm{Higgs}}$ describes the Higgs field,
    $\mathcal{L}_{\mathrm{Yukawa}}$ describes the interaction between the Higgs
    field and fermions,
    $\mathcal{L}_{\mathrm{GF}}$ is a gauge fixing term,
    and $\mathcal{L}_{\mathrm{ghost}}$ is a ghost term.
    The last two terms are required to remove unphysical degrees of freedom
    when gauge fixing the theory.
    All terms in the SM Lagrangian are invariant under local transformations
    of the gauge group (\ref{eqn:SM_gauge}).

    In the following section we will focus on the gauge, fermion, gauge-fixing
    and ghost Lagrangian terms in the framework of QCD as that will be the
    most relevant sector for this thesis. The remaining terms are discussed at
    length in standard reference texts \cite{Peskin:1995ev,Schwartz:2014sze,Romao:2012pq}.


\section{Introduction to quantum chromodynamics}\label{sec:qcd}
    \subsection{The QCD Lagrangian}\label{sec:L_qcd}
    Quantum chromodynamics (QCD) is the sector of the SM that
    describes the strong interaction. 
    QCD is a non-Abelian gauge theory with gauge group
    $\mathrm{SU}(3)_{c}$ where the charge is named colour.
    The gauge and fermion part of the QCD Lagrangian is
    \begin{equation}\label{eqn:L_QCD}
        \mathcal{L}_{\mathrm{QCD}} = -\dfrac{1}{4}F^{a}_{\mu\nu}F^{a, \, \mu\nu}
        + \sum_{f} \bar{\psi}_{i}^{f}(\mathrm{i}\slashed{D}_{ij} - \delta_{ij}m_{f})\psi_{j}^{f} \, ,
    \end{equation}
    where twice repeated indices are summed over.
    The fields $\psi_{i}^{f}$ ($\bar{\psi}_{i}^{f}$) are the fermions (antifermions),
    representing quarks (antiquarks) with flavours $f$: 
    up $u$, down $d$, strange $s$, charm $c$, top $t$, and bottom $b$,
    with masses $m_{f}$. They transform under the fundamental
    representation with indices $i, j \in \{1, 2, 3\}$, named colour indices.
    The gauge fields $A^{a}_{\mu}$, corresponding to gluons, 
    appear in the quark covariant derivative
    \begin{equation}\label{eqn:covariant_deriv}
        (D_{\mu})_{ij} = \delta_{ij}\partial_{\mu} - ig_{\mathrm{s}}T^{a}_{ij}A^{a}_{\mu} \, ,
    \end{equation}
    and the gauge field strength tensor
    \begin{equation}\label{eqn:F_munu}
        F^{a}_{\mu\nu} = \partial_{\mu}A^{a}_{\nu} - \partial_{\nu}A^{a}_{\mu} + g_{\mathrm{s}} f^{abc}A^{b}_{\mu}A^{c}_{\nu} \, .
    \end{equation}
    Gauge fields transform under the adjoint representation, 
    which is an $N_{c}^{2}-1$ dimensional representation, meaning
    the adjoint indices $a,b,c \in \{1,...,8\}$. $T^{a}_{ij}$ are the
    group generators in a fundamental representation. In SU(3) it is
    common to write the group generators as
    \begin{equation}\label{eqn:group_generators}
        T^{a}_{ij} = \dfrac{1}{2}\lambda^{a}_{ij}
    \end{equation}
    where $\lambda^{a}_{ij}$ are the Gell-Mann matrices \cite{Gell-Mann:1962yej}.
    The generators of the group have to obey the Lie algebra
    \begin{equation}\label{eqn:lie_algebra}
        [T^{a}, T^{b}] = \mathrm{i}f^{abc}T^{c} \, ,
    \end{equation}
    where $f^{abc}$ are the structure constants of SU(3).
    The gauge coupling of the group $g_{\mathrm{s}}$
    is a dimensionless free parameter of the theory.
    It is common to use the strong coupling constant
    instead, 
    \begin{equation}\label{eqn:alpha_s}
        \alpha_{\mathrm{s}} = \dfrac{g_{\mathrm{s}}^{2}}{4\pi} \, .
    \end{equation}
    The strong coupling constant, is in fact not constant, but
    is dependent on the energy scale of the process. This
    is due to the process of renormalisation which will be discussed
    in section ...
    The implications of this is that at collider
    experiments where collisions are extremely high energy,
    the strong coupling constant becomes small, a
    property known as asymptotic freedom. A consequence
    of this is that predictions made in QCD can be expressed
    in the form of perturbative expansions in $\alpha_{\mathrm{s}}$.

    To remove unphysical degrees of freedom from the
    theory we need to gauge-fix the theory
    and add a ghost Lagrangian. The gauge-fixing term
    in the $R_{\xi}$ gauges is written as
    \begin{equation}\label{eqn:L_GF}
        \mathcal{L}_{\mathrm{GF}} = -\dfrac{1}{2\xi}(\partial^{\mu}A_{\mu}^{a})^{2} \, ,
    \end{equation}
    where $\xi = 1$ corresponds to the Feynman-'t-Hooft gauge.
    The ghost Lagrangian is most commonly written
    in the Faddeev-Popov procedure as
    \begin{equation}\label{eqn:L_ghost}
        \mathcal{L}_{\mathrm{ghost}} = (\partial_{\mu}\bar{c}^{a})(\delta^{ac}\partial_{\mu} + g_{\mathrm{s}}f^{abc}A^{b}_{\mu})c^{c} \, ,
    \end{equation}
    where $c^{a}$ ($\bar{c}^{a}$) are Faddeev-Popov ghosts (anti-ghosts).
    They are unphysical external states but must be
    included in internal lines for a Lorentz invariant
    perturbative gauge theory.

    \subsection{QCD Feynman rules}\label{sec:qcd_feynman}
    Now that we have written down the QCD Lagrangian,
    we can examine the terms to write down how the
    gauge bosons interact with the fermions. A convenient
    way to do this is with Feynman rules, which can be
    pieced together to form Feynman diagrams. We will see
    that Feynman diagrams are a simple, but powerful, tool
    to systematically build all the possible ways a process
    can occur. This will be discussed further in the
    scattering amplitude section.

    Expanding (\ref{eqn:L_QCD}) and extracting the terms
    that mix the gauge and fermions fields, we get an interaction Lagrangian
    \begin{equation}\label{L_int}
        \mathcal{L}_{\mathrm{int}} = g_{\mathrm{s}}A^{a}_{\mu}\bar{\psi}_{i}^{f}\gamma^{\mu}T^{a}_{ij}\psi_{j}^{f}-g_{\mathrm{s}}f^{abc}(\partial_{\mu}A^{a}_{\nu})A^{b,\,\mu}A^{c,\,\nu} -\dfrac{g_{\mathrm{s}}^{2}}{4}f^{abc}A^{b}_{\mu}A^{c}_{\nu}f^{ade}A^{d,\,\mu}A^{e,\,\nu} \, ,
    \end{equation}
    which we can interpret as follows: the first term
    is an interaction between a gluon, a quark, and an
    anti-quark, the second term is a triple gluon
    self-interaction and the third term is a four gluon
    self-interaction. These mixing terms can be recast
    as Feynman rules as interaction vertices.

    To connect vertices we require propagators, which
    we can read off from the kinetic Lagrangian where
    we collect terms from (\ref{eqn:L_QCD}) and (\ref{eqn:L_GF}), 
    \begin{equation}\label{L_kin}
        \mathcal{L}_{\mathrm{kin}} = -\dfrac{1}{4}(\partial_{\mu}A^{a}_{\nu}-\partial_{\nu}A^{a}_{\mu})^{2} - \dfrac{1}{2\xi}(\partial_{\mu}A^{a}_{\mu})^{2} + \bar{\psi}_{i}^{f}(\mathrm{i}\slashed{\partial}-m_{f})\psi_{i}^{f} \, ,
    \end{equation}
    where the first two terms corresponds to a gluon
    propagator and the third term corresponds to
    a quark propagator.

    {\color{red} Need to write down the Feynman
    rules and draw diagrams. Best to put it in a
    figure that we can refer back to.}

    Note that we have only written down the vertices,
    propagators, and external lines for the physical gluons and quarks,
    ghosts also have associated Feynman rules but we do
    not collect those here. A full list of Feynman rules
    in the SM can be found at Ref. \cite{Romao:2012pq}.

    From these Feynman rules, we identify that each
    quark-antiquark-gluon and triple gluon vertices
    are associated with one power of $\alpha_{\mathrm{s}}$,
    and the four gluon vertex is associated with
    two powers of $\alpha_{\mathrm{s}}$. This relationship
    between vertices and $\alpha_{\mathrm{s}}$ allows
    us to systematically build up Feynman diagrams
    that correspond to a fixed order in $\alpha_{\mathrm{s}}$.
    This point will be elaborated on in Section
    \ref{sec:xs} where we discuss scattering amplitudes
    and how they are related to Feynman diagrams.

\section{Factorisation theorem}\label{sec:factorisation}
    At the LHC, collisions occur between two protons,
    which have constituent quarks and gluons, collectively
    named partons.
    The application of QCD to describe phenomenon
    in these proton-proton collisions rests on the use of
    the factorisation theorem which enables the
    separation of soft and hard energy scales.
    Within the proton, there are composite quarks
    and gluons that are constantly absorbed and
    emitted on a timescale inversely proportional
    to the mass of the proton (or hadron in general).
    The timescale of this fluctuation is much longer
    than the timescale of the interaction between
    a parton and a highly energetic probing parton
    from another proton. This is the scenario
    that occurs at collisions at the LHC.
    During the collision, or the hard scattering,
    the probe is able to interact with a
    parton that is effectively frozen.
    The probing parton knows nothing of the
    proton being probed, except for the fact
    that it collided with a parton carrying
    a fraction of the proton momentum. The
    distribution of partons within a proton
    is process independent and a fundamental
    property of the proton, meaning it can be
    separated from the actual scattering between
    the partons.

    This heuristic argument motivates the form of
    the factorisation equation, where the production of
    particles from a hadronic collision, the hadronic
    cross-section, can be written as
    \begin{equation}\label{eqn:hadronic_cs}
        \sigma_{AB \rightarrow n} = \sum_{a, b}\int_{0}^{1} \mathrm{d}x_{a}\mathrm{d}x_{b} \; f_{a/A}(x_{a},\mu_{\mathrm{F}})f_{b/B}(x_{b}, \mu_{\mathrm{F}}) \; \hat{\sigma}_{ab \rightarrow n}(Q, \mu_{\mathrm{F}}, \mu_{\mathrm{R}}) + \mathcal{O}\left(\dfrac{\Lambda_{\mathrm{QCD}}}{Q}\right)\, ,
    \end{equation}
    where $a$ is the parton from hadron $A$,
    and $b$ is the parton from hadron $B$.
    The sum over initial-state partons $a$ and $b$
    indicates the sum over flavours, which depends
    on the hadron composition in general.
    This factorisation is not exact as indicated by the
    correction term inversely proportional to characteristic
    hard scale $Q$. However, for high energy collisions, where
    $Q \gg \Lambda_{\mathrm{QCD}}$, this term and higher order
    terms are suppressed.
    In fact, the factorisation equation has only been
    proven for a few specific cases \cite{Collins:2011zzd,Collins:1987pm,Collins:1989gx,Amoroso:2022eow}.

    $f_{a/h}(x, \mu)$ are parton distribution functions
    (PDFs) which depend on the momenta fractions $x$ parton $a$
    has with respect to its parent hadron $h$, at a scale $\mu$,
    usually taken to be the factorisation scale $\mu_{\mathrm{F}}$,
    the interface between soft and hard physics.
    The interpretation of PDFs at leading order are
    as probability distributions. $f_{a/h}(x, \mu)$ is
    the probability of finding parton $a$ within $h$
    carrying a momentum fraction $x$ at the energy scale $\mu$.
    PDFs are non-perturbative objects that encapsulate
    the soft effects of the scattering which occur below energy
    $\mu_{\mathrm{F}}$. They are non-perturbative because
    they are determined by fitting to experimental data,
    instead of calculated in perturbation theory.
    For a review of PDFs and how they are determined
    see Ref. \cite{Ethier:2020way,Jimenez-Delgado:2013sma}.
    In practice, these PDF fits are accessed through the
    LHAPDF interface \cite{Buckley:2014ana} which provides
    PDF sets from multiple working groups \cite{PDF4LHCWorkingGroup:2022cjn,Hou:2019efy,Bailey:2020ooq,NNPDF:2021njg}.
    The evolution of PDFs between factorisation scales is possible
    through the use of the Dokshitzer-Gribov-Lipatov-Altarelli-Parisi (DGLAP)
    equations \cite{Dokshitzer:1977sg,Gribov:1972ri,Lipatov:1974qm,Altarelli:1977zs}.

    The momenta fractions, $x$, also link the
    squared centre-of-mass energies of the hadronic collision,
    denoted as $s$, to the partonic equivalent as
    \begin{equation}\label{eqn:E_cm_partonic}
        \hat{s} = x_{a}x_{b}s \, .
    \end{equation}

    $\hat{\sigma}_{ab \rightarrow n}(Q, \mu_{\mathrm{F}}, \mu_{\mathrm{R}})$
    is the partonic cross-section which describes the interaction of
    partons $a$ and $b$ scattering into $n$ particles,
    where the scattering occurs at energy scale $Q$,
    which is often taken to the $\hat{s}$.
    Notice that the partonic cross-section depends on
    both the factorisation scale, $\mu_{\mathrm{F}}$ and
    the renormalisation scale, $\mu_{\mathrm{R}}$
    (see \ref{chapter:qcd}).
    {\color{red} Need to mention how these scales
    are unphysical but there is a residual dependence
    on them at fixed-order calculations. Scale
    variations are a conventional way to determine
    the effects of higher order terms.
    Renormalisation will be discussed in Chapter 2
    when divergences are introduced.}

    The discussion so far has been focused hadron-hadron
    initiated scattering, however, a similar argument
    can be made about electron-positron annihilation,
    where the electron-positron annihilates to form a
    photon ($Z$ boson) which fluctuates into a quark-antiquark
    pair. For a sufficiently high energy collision,
    these interactions can be calculated in perturbation
    theory.
    Since electrons and positrons are not composite
    particles, there is no requirement for PDFs,
    therefore (\ref{eqn:hadronic_cs}) simplifies
    to just the partonic cross-section, which
    will be computed in perturbation theory.
    
    The details of how $\hat{\sigma}_{ab \rightarrow n}$
    is computed is discussed in the next section.

\section{Scattering amplitudes and cross-sections}\label{sec:xs}
    As mentioned in the previous section, at
    collider experiments we typically collide
    two beams, consisting of bundles of energetic
    particles, and analyse the resulting products
    of these collisions. The predictions we make
    from our theoretical model are the partonic
    cross-sections. To arrive at an expression
    for the partonic cross-section, we need to
    first consider the hard scattering of particles.
    To model the hard collisions in
    a collider experiment, we look at the specific
    case of two particles colliding into $n$ particles.
    Scattering amplitudes are used to mathematically
    describe these scattering processes.
    For an initial-state
    $|i\rangle$ and final-state $|f\rangle$
    the scattering amplitude can be written as
    \begin{equation}\label{eqn:scattering}
        \langle f | S | i \rangle \, ,
    \end{equation}
    where the scattering matrix, or $S$-matrix
    can be decomposed into an identity matrix
    and a transfer matrix $\mathcal{T}$,
    \begin{equation}\label{eqn:S_matrix}
        S = \mathbbm{1} + \mathrm{i}\mathcal{T} \, .
    \end{equation}
    The $S$-matrix encodes all the information
    about how the initial-state will evolve
    into the final-state over time. By writing
    the $S$-matrix in this way, all the interactions
    are separated into the transfer matrix $\mathcal{T}$,
    as the identity matrix describes the free theory.
    By imposing a momentum conserving constraint
    on the $S$-matrix, the matrix element,
    $\mathcal{M}$, is defined in the following expression
    \begin{equation}\label{eqn:matrix_element}
        \langle f | \mathcal{T} | i \rangle = (2\pi)^{4} \delta^{4}\left(p_{a} + p_{b} - \sum_{f=1}^{n} p_{f}^{\mu}\right) \mathcal{M} \, ,
    \end{equation}
    where we take $p_{a}$ and $p_{b}$ to be
    the momenta of the two colliding
    particles in the initial-state, and $p_{f}$ to be the
    momenta of the $n$-particle final states.
    The transition probability of this scattering process
    is the modulus squared of the scattering amplitude
    \begin{equation}\label{eqn:S_prob}
        P = |\langle f | S | i \rangle|^{2} \propto | \langle f | \mathcal{M} | i \rangle |^{2}
    \end{equation}
    where $|\langle f | \mathcal{M} | i \rangle|^{2}  \equiv |\mathcal{M}|^{2}$
    is the matrix element squared \footnote{from henceforth we refer to the matrix
    element squared as matrix element}.
    
    Since detectors at collider experiments
    have limited resolution, single scattering
    events are not measured. Instead we consider
    the cross-section, $\sigma$, as introduced in
    (\ref{eqn:hadronic_cs}). The cross-section is a property
    of the particles being scattered, and is independent
    of the way the experiment is carried out.
    Since the hadronic cross-section factorises
    the soft energy effects into the non-perturbative PDFs,
    which are determined once and for all
    processes, the object of interest now becomes the
    partonic cross-section, $\hat{\sigma}$, which
    encodes the scattering information of the specific
    process considered.

    Given that cross-sections are measurable
    quantities, the partonic cross-section directly
    relates the matrix elements which we calculate in our QFT,
    to measurable observables at experiments.
    In practice, it is more useful to consider the
    differential cross-section, $\mathrm{d}\hat{\sigma}$,
    where the cross-section can be differential
    in quantities such as energies and angles.
    This is because the final-state particles
    will be produced with a range of, for example,
    energies and momenta so the cross-section would
    be vanishing for an exact momentum configuration.

    The differential partonic cross-section can be written as
    \begin{equation}\label{eqn:dsigma}
        \mathrm{d}\hat{\sigma} = \dfrac{1}{\mathcal{F}}|\mathcal{M}_{2 \rightarrow n}|^{2} \mathrm{d}\Phi_{n} \, .
    \end{equation}
    $\mathcal{F}$, the flux factor is
    \begin{equation}\label{eqn:flux}
        \mathcal{F} = 4\sqrt{(p_{a}p_{b})^{2} - m_{a}^{2}m_{b}^{2}} \, ,
    \end{equation}
    which in the massless limit reduces to
    \begin{equation}\label{eqn:massless_flux}
        \mathcal{F} = 2(p_{a} + p_{b})^{2} = 2\hat{s} \, .
    \end{equation}
    It is common to use the massless limit because
    the centre-of-mass energy of a collider
    experiment is much greater than the mass
    of the colliding particles.

    $\mathrm{d}\Phi_{n}$ is the Lorentz-invariant
    phase-space which contains all the possible
    configurations of the $n$-particle final state.
    Absorbing the momentum conserving $\delta$-function
    from (\ref{eqn:matrix_element}),
    we can write it as
    % \begin{equation}\label{eqn:dlips}
    %     \mathrm{d}\Phi_{n} = (2\pi)^{d}\delta^{(d)}\left(p_{a} + p_{b} - \sum_{f} p_{f}^{\mu}\right) \prod_{f=1}^{n} \dfrac{\mathrm{d}^{d}p_{f}}{(2\pi)^{d-1}}\delta(p_{f}^{2} - m_{f}^{2})\theta(E_{f}) \, ,
    % \end{equation}
    \begin{equation}\label{eqn:dlips_4d}
        \mathrm{d}\Phi_{n} = (2\pi)^{4}\delta^{4}\left(p_{a} + p_{b} - \sum_{f=1}^{n} p_{f}^{\mu}\right) \prod_{f=1}^{n} \dfrac{\mathrm{d}^{4}p_{f}}{(2\pi)^{3}}\delta(p_{f}^{2} - m_{f}^{2})\theta(E_{f}) \, ,
    \end{equation}
    where the second $\delta$-function restricts
    the final state particles to be on mass-shell
    (on-shell). The step function selects only
    the positive energy solution for this constraint.
    This gives an intuitive picture of the partonic
    cross-section as the probability of a $2\rightarrow n$
    process occurring, summed over all the possible
    valid final-state configurations.

    From (\ref{eqn:S_prob}), it becomes clear
    that the matrix element is the object which
    relates back to the Lagrangian of the theory
    as it encodes the interactions of the particles.
    We saw in Section \ref{sec:qcd_feynman} that
    interactions could be codified into Feynman
    rules which are pieced together to construct
    Feynman diagrams. In this picture, matrix elements
    are exactly the sum over all Feynman diagrams
    \footnote{technically the interference of
    Feynman diagrams as we are working on the level
    of $|\mathcal{M}|^{2}$}
    for a particular process. However, there are
    an infinite number of Feynman diagrams you can
    draw as there is no constraints on how the
    vertices and propagators can be connected.
    Fortunately because $\alpha_{\mathrm{s}}$
    is small in high energy collisions, we can expand
    the matrix element as a perturbative series in
    $\alpha_{\mathrm{s}}$ to write
    \begin{equation}\label{eqn:matrix_element}
        |\mathcal{M}|^{2} =  \alpha_{\mathrm{s}}^{m} |\mathcal{M}|^{2}_{\mathrm{LO}} + \alpha_{\mathrm{s}}^{m+1} |\mathcal{M}|^{2}_{\mathrm{NLO}} + \alpha_{\mathrm{s}}^{m+2} |\mathcal{M}|^{2}_{\mathrm{NNLO}} + \mathcal{O}(\alpha_{\mathrm{s}}^{m+3}) \, ,
    \end{equation}
    where $m$ corresponds to the powers of $\alpha_{\mathrm{s}}$
    in the simplest Feynman diagrams you can draw
    for the process of interest.
    Since each term of this expansion is at a fixed-order
    in $\alpha_{\mathrm{s}}$, it is possible to systematically
    compute the diagrams which contribute at the given order of $\alpha_{\mathrm{s}}$.
    The set of diagrams that contribute at the lowest
    order in $\alpha_{\mathrm{s}}$ are $|\mathcal{M}|^{2}_{\mathrm{LO}}$,
    where LO stands for leading order. The next term
    in the expansion corresponds to the leading order
    diagrams with an additional loop, or external leg,
    which contributes an extra factor of $\alpha_{\mathrm{s}}$.
    This term is dubbed NLO for next-to-leading order
    in $\alpha_{\mathrm{s}}$, and the following term
    next-to-next-to-leading order has again an additional
    loop or leg. In general, we have terms
    N$^{k}$LO where each additional power of $\alpha_{\mathrm{s}}$
    increases complexity of the computations, however,
    because the expansion $\alpha_{\mathrm{s}}$
    is small, each additional term contribute less.
    This means that the first terms dominate the expansion,
    and it is sufficient to terminate the series after
    a few terms to reach an acceptable level of accuracy.
    A further discussion of this point will be given
    in Chapter~\ref{chapter:qcd} where we break down
    the computation of each term in (\ref{eqn:matrix_element}).

    In summary, the partonic cross-section of a collision
    reduces to the computation of the matrix elements
    up to a fixed-order in the coupling expansion,
    which is then integrated over the valid phase-space of
    final-state particle configurations. For hadronic
    collisions the partonic cross-section also needs to
    be convoluted with the PDFs to obtain the hadronic
    cross-sections.

    With the introduction of matrix elements and
    cross-sections complete, we will move the discussion onto
    Monte Carlo event generators (MCEG). These event
    generators are multipurpose tools that
    incorporate many aspects of the theory discussed
    so far (and to be discussed) for practical use.
    Most importantly, they are
    used to compare theoretical predictions to
    experimental measurements.

\section{Monte Carlo event generator overview}
\subsection{Event unweighting}
\begin{itemize}
    \item MC generators are the main tools that
    people use to compare with experimental data.
    There are many groups that have their own tool.
    \item The usage of MC tools came about because
    the multi-dimensional phase-space integrals
    are too complex to do analytically.
    \item This doesn't mean that they only do
    the integrals, the event generators basically
    contain all the tools from phase-space generation
    to matrix element evaluation, to parton showers
    to hadronisation.
    \item Have a look at Katherina thesis to see
    how much I can skip because only really want to
    introduce unweighting. Probably along the lines
    of event weight is statistically sampled vs
    what they measure in real life. Therefore we need
    to unweight events.
\end{itemize}
\end{document}